\section{Discussion}
  \label{sec:twophoton_discussion}

  The theoretical model built up in this chapter of light interacting with the
  manifolds of the rubidium atom, including degenerate hyperfine structure,
  allows us to investigate transfer of population from the ground state to
  excited levels. By going beyond the weak probe approximation we can properly
  simulate the atom-light interaction at the high beam powers used in the
  motivating experiments, far beyond saturation intensity.

  The model proceeds by solving the Lindblad master equation for density matrix
  elements covering the basis of hyperfine sublevels over a time period of
  \unit[2]{$\mu $s} representing the average transit time of atoms in the beam.
  The system is then solved over a range of detunings covering the \textsc{d2}
  lines to simulate a scan of the probe beam.

  Agreement with results of the \textit{ElecSus} model for low intensities,
  where the weak probe approximation is valid, gives us confidence in the model
  to take it to higher intensities.At intensities where the power broadening is
  small enough to be compatible with the experimental data (on the order of $I =
  \unit[100]{W/cm^2}$) the population to the $6$P manifold via two-photon
  excitation of the $5$D state on the order of $10^{-11}$. In the
  \unit[2]{mm}-long cell at the higher densities of
  $\unit[7\times10^{14}]{cm^{-3}}$, on the order of $10^{8}$ atoms will be
  interacting with the beam. The results obtained at those higher intensities
  thus suggest that two-photon excitation cannot be responsible for the observed
  fluorescence.

  On the other hand, at higher intensities where the transfer of
  population might be considered significant, power broadening would completely
  wash out the spectral features observed.

  We can thus rule out two-photon excitation as a mechanism for the
  fluorescence. One possible mechanism which might next be considered is energy
  pooling, whereby one atom transfers to a higher energy state in an elastic
  collision\cite{Namiotka1997,Bearman1978}. Another is a cooperative
  effect\cite{Bettles2016,Bettles2015}, whereby the dipole-dipole interaction
  between atoms in proximity in an optical field may increase dramatically when
  they are closer than a critical distance $\lambda/2\pi$, where $\lambda$ is
  the resonant wavelength of the field.

  The \textit{OpticalBloch} simulation package developed for this calculation
  for solving for the time-evolution of atomic systems, including the full
  hyperfine structure and all the relevant angular momentum factors in
  calculating transition dipole matrix elements, is available for future studies
  in many applications involving vapours of alkali metals beyond the weak probe,
  and can easily be extended to consider the including of the Zeeman effect due
  to the application of magnetic fields.
