\section{Introduction}
  \label{sec:twophoton_intro}

    In keeping with the theme of atom-light interactions beyond linear regimes,
    in this chapter we present a theoretical study of the interaction of a high-
    intensity beam with a thermal vapour of rubidium. Specifically, we
    investigate fluorescence from high-level states, observed in scans across
    the \textsc{d2} spectral lines. Recent experimental work\cite{Weller2013}
    found dramatic enhancement of this fluorescence above a critical density,
    indicating an increase in population transfer from the ground state to the
    higher $5\rm{d}$ states.

    The high intensity of the beam makes invalid any recourse to the weak probe
    approximation and necessitates the consideration of the significant
    mechanisms of power broadening and hyperfine optical pumping. We must also
    in our model take into account Doppler broadening due to the motion of
    thermal atoms in the cell.

    The combination of hyperfine pumping and the finite transit time of 
    non-stationary atoms in the beam also prevents us from simply taking the 
    steady-state atomic response: we must solve for quantum evolution of the 
    atomic system via the density matrix. The computational model here developed
    to study the system dynamics uses an exponential series solver for the
    Lindblad master equation (\ref{eqn:lindblad}) over the many-level basis of
    hyperfine angular momentum sublevels, along with parallel computation across
    the range of detunings.
