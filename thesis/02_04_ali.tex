\section{Interaction of Light with Atoms}
  \label{sec:propagation_ali}

  We have seen in the weak field regime how the linear susceptibility relates to
  the absorptive and dispersive response of the medium. But we have not yet
  determined how that susceptibility relates to the properties of the atomic
  ensemble. Beyond the linear regime, the susceptibility is not a good
  descriptor for the response of the medium, and analytic expressions for the
  field propagation are not available. In the nonlinear regime, we will need to
  follow the dynamics of the atoms to determine how the field will propagate
  through the medium.

  For the dynamics of atom-light interaction to be properly considered the
  system must be treated as an open quantum system. While the processes of
  absorption and stimulated emission (of photons from and to the applied field)
  can be described within a closed quantum system, the process of spontaneous
  decay due to interaction with vacuum fluctuations surrounding an atom cannot.
  As such, the time evolution is described by the Lindblad master equation
    \begin{equation}\label{eqn:lindblad}
      \ii \hbar \frac{\partial \rho}{\partial t} = [\mathcal{H}, \rho] + 
        \mathcal{L}\left\{ \rho \right\}.
    \end{equation}
  The background of this equation and the conditions under which it is valid are
  discussed in appendix \ref{apx:qu_dyn}. It constitutes a set of differential
  equations to be solved: one for each of the density matrix elements.

  \subsection{Spontaneous Decay of Excited States}

    As described in equation (\ref{eqn:lindblad_op}), coupling to the
    environment is implemented via the Lindblad superoperator $\mathcal{L}$
    which is completely defined by a finite set of collapse operators. In the
    case of spontaneous emission due to interaction with vacuum fluctuations,
    these are defined as
    \begin{equation}
      C_{ij} = \sqrt{\Gamma_{ij}} \Ket{i} \Bra{j}.
    \end{equation}
    where $\Gamma_{ij} = 1/\tau_{ij}$ and $\tau_{ij}$ is the stochastic rate at
    which electrons spontaneously decay from a higher state $\Ket{j}$ to a lower
    state $\Ket{i}$. The quantity $\Gamma_{ij}$ is known as the natural
    linewidth for the specific transition $\Ket{i} \rightarrow \Ket{j}$, for
    reasons that will be clarified in section \ref{sec:propagation_twolevel}.

  \subsection{The Interaction Hamiltonian}

    A single-electron (or hydrogenic) atom has a positively charged nucleus and
    a negatively charged electron, both of which will interact with an applied
    electromagnetic field. At optical wavelengths, however, the interaction with
    the nucleus is negligible\cite{bransden2003physics}, so we focus our
    attention on the electron.

    Without interaction with an external field, the bare atomic Hamiltonian is
    given by
    \begin{equation}
      \mathcal{H}_0 = \frac{\mathbf{p}^2}{2 m_e} + V(r)
    \end{equation}
    where $m_e$ is the mass of the electron, $\mathbf{r}$ and $\mathbf{p} = \ii
    \hbar \mathbf{\nabla}$ are its position and momentum operators, and $V(r)$
    is the spherical atomic potential.

    The non-relativistic Hamiltonian of the electron interacting with an applied
    classical electric field $\mathbf{E}$, in the Coulomb gauge, may be written
    \begin{equation}
        \mathcal{H} = \mathcal{H}_0 + \mathcal{H}_I
    \end{equation}
    where the interaction Hamiltonian term $\mathcal{H}_I$ describes the
    coupling of the atomic dipole to the field.\cite{bransden2003physics}  The
    problem of the atom-light interaction is then one of calculating the matrix
    elements of $\mathcal{H}_I$ as a perturbation to the eigenstate basis of
    $\mathcal{H}_0$.

    We will consider a monochromatic field with angular frequency $\omega$ and
    wavenumber $\mathbf{k}$ (the analysis extends to multi-chromatic fields,
    which we will consider in chapter \ref{chp:nonlinear}). We may write the
    field as
    \begin{equation}\label{eqn:envelope_carrier}
      \mathbf{E}(\mathbf{r}, t) = \hat{\mathbf{x}} 
      \left[ \tfrac{1}{2} \mathcal{E}(t) 
      \mathrm{e}^{\mathrm{i}(\mathbf{k} \cdot \mathbf{r} - \omega t)} + 
      \tfrac{1}{2} \mathcal{E}^*(t) \mathrm{e}^{-\mathrm{i}(\mathbf{k} \cdot 
      \mathbf{r} - \omega t)} \right]
    \end{equation}
    where $\hat{\mathbf{x}}$ is the unit polarisation vector and 
    $\mathcal{E}(t)$ is the field amplitude.    

   We simplify calculation of matrix elements by making the exponential
   expansion
    \begin{equation}
      \ee^{\ii\mathbf{k}\cdot\mathbf{r}} \approx 1 + 
      (\ii \mathbf{k}\cdot\mathbf{r}) + 
      \frac{1}{2!} ( \ii \mathbf{k}\cdot\mathbf{r})^2 + \dots 
    \end{equation}
    and neglecting all but the first term, unity. This \textit{electric dipole
    approximation}\cite{grynberg2010introduction} represents neglecting the
    spatial dependence of the field over the extent of the atom, and is
    justified as the electronic wavefunction is on the order of the Bohr radius
    at $\unit[10^{-10}]{m}$ and the optical carrier wavelength $\lambda =
    2\pi/k$ is on the order of $\unit[10^{-7}]{m}$. The approximation may
    equivalently be derived as truncating a multipole expansion of the
    interaction at the dipole term.\cite{cohen1992atom}

    Applying the electric dipole approximation, we may write the interaction Hamiltonian term as 
    \begin{equation}
      \mathcal{H}_I = -e \mathbf{r} \cdot \mathbf{E} 
                    = -\mathbf{d} \cdot \mathbf{E}
    \end{equation}
    where analogous with a classical dipole moment, $\mathbf{d}$ is the
    electric dipole operator. And we may take the field out of the spatial
    integral implicit in calculating the matrix elements between two bare atom
    eigenstates $\Ket{a}$ and $\Ket{b}$,
    \begin{align}
      \Bra{a} \mathcal{H}_I \Ket{b} &= -
        \left [ \tfrac{1}{2} \mathcal{E}(t) 
        \mathrm{e}^{- \mathrm{i} \omega t} +
        \tfrac{1}{2} \mathcal{E}^*(t) \mathrm{e}^{\mathrm{i} \omega t} \right]
        \Bra{a} \hat{\mathbf{x}} \cdot e \mathbf{r}
        \Ket{b} \nonumber \\
        &= - \left [ \tfrac{1}{2} \mathcal{E}(t) 
        \mathrm{e}^{- \mathrm{i} \omega t} +
        \tfrac{1}{2} \mathcal{E}^*(t) \mathrm{e}^{\mathrm{i} \omega t} \right] 
        d_{ab}
    \end{align}
    where $ d_{ab}$ is then the matrix element of the electric dipole operator
    $\mathbf{d}$ projected on the polarisation direction of the electric field.
    The crux of the problem is then in calculating (or looking up) dipole matrix
    elements for the eigenstates of a given system.

  \subsection{Dipole Matrix Elements and Parity}

    We can show that the diagonal matrix elements of $\mathbf{d} = e\mathbf{r}$
    are zero by making a parity argument. We define the parity operator $\Pi$ as
    the unitary operator (i.e. $\Pi^\dagger\Pi = 1$) that flips the sign of the
    position operator $\mathbf{r}$ via
    \begin{equation}
      \Pi \mathbf{r} \Pi^\dagger = -\mathbf{r}.
    \end{equation}

    Operating with $\Pi$ on the right of both sides shows that the anticommutator
    $\{ \Pi, \mathbf{r} \} = \Pi \mathbf{r} + \mathbf{r} \Pi = 0$ and thus the
    matrix elements vanish
    \begin{equation}
      \left< i \right| \{ \Pi, \mathbf{r} \} \left| j \right> = 
      \left< i \right| \Pi \mathbf{r} + \mathbf{r} \Pi \left| j \right> = 0 
    \end{equation}
    for any states $\Ket{i}, \Ket{j}$. Now $\Pi$ commutes with $\mathcal{H}$,
    and so has the same eigenstates, so we have eigenvalues $\pi_i, \pi_j$ such
    that $\Pi \Ket{i} = \pi_i \Ket{i}$ and $\Pi \Ket{j} = \pi_j \Ket{j}$. Thus
    we can write
    \begin{equation}
      \left< i \right| \Pi \mathbf{r} + \mathbf{r} \Pi \left| j \right> = 
      (\pi_i + \pi_j) \left< i \right|  \mathbf{r} \left| j \right>.
    \end{equation}
    The right hand side can only be zero if $\pi_i + \pi_j$ is zero or if the
    matrix element is. Now, as $\Pi^2 = 1$, the eigenvalues $\pi = \pm1$. So for
    the diagonal matrix elements, $\pi_i + \pi_i$ can't be zero and $\pi_j +
    \pi_j$ can't be zero so it must be that $\left< i \right|\mathbf{r} \left| i
    \right> = \left< j \right|\mathbf{r} \left| j \right> = 0$. The off-diagonal
    matrix elements $\left< i \right|\mathbf{r} \left| j \right>$ are  non-
    vanishing if the states have opposite parity such that $\pi_i = -\pi_j$.

  \subsection{Atomic Coherence and Polarisation}

    We introduced the polarisation $P$ in section \ref{sec:propagation_deriving}
    as the cumulative effect of charge separation induced on individual atoms
    and defined it in \ref{sec:propagation_susc} as the dipole moment per unit
    volume.  In terms of atomic observables, the polarisation at a distance $z$
    through the medium at time $t$ may therefore be written as the expectation
    value of the scalar dipole operator for those atoms
    \begin{equation}
      P(z,t) = N(z) \langle \mathrm{d}(z,t) \rangle
    \end{equation}

    where $N(z)$ is the number density (atoms per unit volume) of the medium,
    which may in general be a function of propagation distance $z$, for example
    in an atom cloud shaped by the geometry of a magneto-optical
    trap\cite{Adams1997}, or constant for a thermal cell in thermal equilibrium.

    The expectation value of an observable for a system in a pure or mixed state
    represented by a density matrix $\rho$ is defined in equation
    (\ref{eqn:density_matrix_exp_value}), such that we may write
    \begin{equation}
      P(z,t) = N(z) \Tr{ \left[ \rho \mathrm{d}(z,t) \right] }.
    \end{equation}
    As we know from the above parity argument that the diagonal matrix elements
    of the dipole operator are zero, we may then write $P$ directly in terms of
    the off-diagonal elements and the atomic coherences, via
    \begin{equation}
      P(z,t) = N(z) \sum_{i \ne j}{\left[ d_{ij} \rho_{ij}(z,t) + 
                                 d_{ji} \rho_{ji}(z,t) \right] }.
    \end{equation}

    Now in order to relate this to the slowly-varying envelope $\mathcal{P}$, we
    need to rotate the density matrix elements, via
    \begin{align*}
      \rho_{ij} &= \tilde{\rho}_{ij} ~ \ee^{\ii (k z - \omega t)} \\
      \rho_{ji} &= \rho^*_{ij} = \tilde{\rho}^*_{ij} ~ \ee^{-\ii (k z - \omega t)}
    \end{align*}
    where tilde-notated variables $\tilde{\rho}_{ij}$ are slowly-varying density
    matrix elements. Dropping the tilde notation, we then derive an expression
    for the slowly-varying polarisation envelope in terms of the atomic
    coherences
    \begin{equation} 
      \mathcal{P}(z,t) = N(z) \sum_{i \ne j} d_{ij} \rho_{ij}(z,t)
      \label{eqn:polarisation_coherences}
    \end{equation} 
    which we may substitute into the propagation equation
    (\ref{eqn:fo_mwe}).

    Note that in our discussion of polarisation in this section we have made no
    reference to the susceptibilities $\chi^{(j)}(t)$. These are implicit in the
    density matrix coherences. This analysis is valid for any general nonlinear
    form of polarisation as expressed in equation (\ref{eqn:gen_polarisation})
    if we can determine the evolution of atomic states from the Lindblad
    equation (\ref{eqn:lindblad}). 

  \subsection{Thermal Atoms}

    The above analysis for the atom-light interaction is appropriate for
    stationary (\ie ultracold) atoms but must be modified for thermal atoms due
    to the averaging effect of atomic motion.\cite{Icsevgi1964}

    An atom moving with a velocity component $v$ in the $z$-direction will
    interact with a Doppler-shifted field frequency $\omega - k v$.  This shift is effected over a \textsc{1d} Maxwell-Boltzmann probability distribution function of velocity\cite{foot2005atomic, Gea-Banacloche1995}
    \begin{equation}
      f(v) = \frac{1}{u \sqrt{\pi}} \ee^{-(k v/u)^2}
      \label{eqn:max_boltz}
    \end{equation}
    where the thermal width $u = k v_w$. Here $k$ is again the wavenumber of the
    quasi-monochromatic field and $v_w = 2 k_B T/m$ is the most probable speed
    of the Maxwell-Boltzmann distribution for a temperature $T$ and atomic mass
    $m$. As is usual, $k_B$ represents the Boltzmann constant.

    To include this Doppler effect in the field propagation equations
    (\ref{eqn:fo_mwe}), we replace the atomic coherence factor by an
    integral over a convolution of $f(v)$, with the atomic coherence now a
    function of velocity, so that
    \begin{equation}
      \mathcal{P}(z,t) = N \sum_{i \ne j} d_{ij} \int_{-\infty}^{\infty} 
                        \rho_{ij}(z,t; v) f(v) {\dd v}.
      \label{eqn:polarisation_doppler}
    \end{equation}
    This velocity-dependent $\rho_{ij}(z,t; v)$ represents the atomic coherence
    resulting from interaction with a field at the Doppler-shifted frequency
    $\omega - k v$.

    The result of the inclusion of thermal effects is a broadening of absorption
    resonance widths, a familiar concern in spectroscopy. We will consider
    example spectral profiles in section \ref{sec:propagation_twolevel}.

  \subsection{Shifting to the Speed-of-Light Reference Frame}

    To solve the propagation equation (\ref{eqn:fo_mwe}) as a boundary value
    problem, it is useful to introduce co-moving variables $\zeta = z$ and $t' =
    t - z/c$. This is equivalent to using a reference frame that moves with the
    speed of light across the medium.\cite{Icsevgi1964} We then have
    \begin{equation}
    \frac{\partial}{\partial \zeta} = \frac{\partial z}{\partial \zeta} \frac{\partial}{\partial z} +  \frac{\partial t}{\partial \zeta} \frac{\partial}{\partial t} = \frac{\partial}{\partial z} + \frac{1}{c} \frac{\partial}{\partial t}      
    \end{equation}
    so that
    \begin{equation}\label{eqn:comoving_mwe}
      \frac{\partial}{\partial z} \mathcal{E}(z,t') = 
        \mathrm{i} \frac{k}{2 \epsilon_0}
        N(z) \sum_{i \ne j} d_{ij} \int_{-\infty}^{\infty} 
          \rho_{ij}(z,t; v) f(v) {\dd v}. 
    \end{equation}
    In this reference frame we see that the propagation equation for the field
    is now a differential equation only in $z$.

  \subsection{A Recap}

    At this point we have derived a set of coupled partial differential
    equations describing the dynamics of the atomic density operator
    (\ref{eqn:lindblad}) and the propagation of the electric field envelope
    (\ref{eqn:comoving_mwe}).

    These coupled equations can be integrated numerically for a given set of
    boundary conditions defining the input profile of the electric field and the
    initial state of the atoms. The integration proceeds via the following
    recipe, which must be repeated in a self-consistent manner:
    \begin{enumerate}
      \item Solve the Lindblad master equation for the quantal dynamics of the
        atomic density matrix over time $t'$.
      \item Average the Maxwell-Boltzmann probability distribution over velocity
        $v$.
      \item Solve the Maxwell wave equations for propagation of the 
        electromagnetic field over space $z$.
    \end{enumerate}

    The description of the specific numerical algorithms used for integration,
    along with details of the Python code used to implement these algorithms, is
    given in appendices \ref{apx:ob_eqns} and \ref{apx:mb_eqns}. We will make
    use of these methods to solve the \textsc{mb} equations for various systems
    of interest throughout the chapters of this thesis.

