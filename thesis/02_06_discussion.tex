\section{Discussion}
  \label{sec:propagation_discussion}

    In this chapter we started from the Maxwell equations and derived a
    propagation equation for monochromatic light, using the slowly varying
    envelope approximation. We introduced the linear and nonlinear optics, and
    for the linear regime, for weak fields, the useful concept of susceptibility
    and how this relates to the absorptive and dispersive response of the
    medium.

    Going beyond the linear regime, we need to follow the quantal dynamics of
    the atomic density matrix, which we do with the Lindblad master equation. We
    defined the interaction Hamiltonian within the electric dipole
    approximation, and how the polarisation of the medium can be derived from
    atomic coherences. For thermal atoms this coherence must be averaged over a
    Maxwell-Boltzmann distribution. We discussed algorithms (presented in
    appendix \ref{apx:mb_eqns}) for integrating the propagation equations
    numerically for nonlinear propagation.

    We presented results for a two-level system in the linear regime. In doing
    so we used spectral analysis to compare with the analytic results derived in
    this regime. The good agreement in these results gives us confidence in the
    numerical methods. In the next chapter, we will employ the model in
    considering nonlinear pulse propagation in two-level and three-level
    systems.
