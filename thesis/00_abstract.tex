  % \addcontentsline{toc}{chapter}{Abstract}
  \thispagestyle{empty}
  \vspace{-1cm}
  \begin{center}
  {\huge \textbf{\thesistitle}}
  \\
  \vspace{0.8cm} {\Large \textbf{\myname}\\\rule{7cm}{0.2mm}}
  \end{center}
  \vspace{0.6cm}
  {\large \textbf{Abstract}}

  \begin{small}
  This thesis presents theoretical models and results of numerical simulations
  describing the propagation of optical pulses through dense, thermal atomic
  vapours. In particular we investigate the nonlinear effects of optical
  solitons due to self-induced transparency (\textsc{sit}) in two-level systems,
  optical simultons in V-type three-level systems and electromagnetically
  induced transparency (\textsc{eit}) in $\Lambda$-type systems, including the
  storage and retrieval of dark-state polaritons.

  An investigation is made into two-photon excitation of the $5$D states of
  rubidium in a high-intensity beam including the hyperfine structure of the
  relevant atomic levels. Decay from these states to the $6$P manifolds is ruled
  out as a cause of experimentally observed fluorescence due to the amount of
  power broadening associated with intensities necessary to provide any
  significant level of population in these highly excited states.

  We combine the nonlinear effects of optical solitons and \textsc{eit} to
  explain experimentally-observed steepened pulses in a V-type system in a
  micron-length cell. We explain the behaviour as the early formation of a
  simulton pulse drawn from a \textsc{cw} probe field by a strong coupling
  pulse, due to coherent population trapping. We predict that in a longer cell
  it may be possible to facilitate propagation of matched pulses, even when the
  transitions in the system have different propagation coefficients, as long as
  decoherence from collision broadening can be controlled. The fact that weak
  pulses can propagate with this scheme suggests an approach to achieving
  transparent propagation of single or few photon pulses distinct from, but
  related to, both \textsc{sit} and \textsc{eit}.

  \end{small}

  \newpage