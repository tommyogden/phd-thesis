\section{Discussion}
  \label{sec:polaritons_discussion}

  The \textsc{eit} technique provides an alternative method to \textsc{sit} for
  transmission of light in an optically dense medium, with more control over its
  associated effects such as slow light and pulse compression. We considered the
  transparency effect by looking at the time-evolution of the atomic density
  matrix elements and the steady state lineshape in the weak probe
  approximation.

  By transforming to the \textsc{cpt} basis we were able to understand
  \textsc{eit} as a coherent effect based upon population of the dark-state
  superposition. Introducing the polaritons quasiparticle allows us to
  understand the basis of this propagation and presents the possibility of
  storing and retrieving light pulses.

  The ability to `stop' light is clearly interesting from a purely scientific
  perspective, but the fact that information encoded in the pulse can be
  reversibly transferred to long-lived spin waves as important applications.
  Significantly, it may be shown that the dark-state polariton picture also
  holds for quantised light fields, such that individual photon wave-packets can
  be stored and retrieved\cite{Fleischhauer2000}. This provides a mechanism for
  quantum memory, a key requirement for quantum information
  processing\cite{Lvovsky2009,Zhao2008}. The high fidently of the \textsc{eit}
  memory scheme compares favourably with other proposals such as cavity
  \textsc{qed} and photon echo techniques\cite{Moiseev2011,Kurnit1964}.

  Finally, by comping a probe transition to Rydberg states in a $\Xi$-type
  system, polaritons can be made to interact due to strong dipole-dipole
  interactions between such highly excited states.\cite{Maxwell2013}
