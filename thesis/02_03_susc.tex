\section{Linear Optics, Susceptibility and Refractive Index}
  \label{sec:propagation_susc}

  % \subsection{Intro}

  The effect of an electric field $\mathbf{E}$ applied on a single neutral atom
  is to separate the positively charged core (which moves in the direction of
  the field $\mathbf{E}$) and the negatively charged electron cloud (which moves
  in the opposite direction $-\mathbf{E}$) such that a dipole is induced on the
  atom parallel to the field. For an atomic vapour subject to a field, dipoles
  will be induced on many atoms. The cumulative effect is that the medium is
  polarised, and we define $\mathbf{P}$ as the dipole moment per unit volume.

  \subsection{Susceptibility}

    In general, the instantaneous polarisation induced by the field at a time
    $t$ is some function of the input field, which we may write as a power
    expansion\cite{boyd2008nonlinear} in $\mathbf{E}$,
    \begin{equation}\label{eqn:gen_polarisation}
      \mathbf{P}_{\mathrm{inst}}(t) = \varepsilon_0 \left
          [ \chi^{(1)}(t) \mathbf{E}(t) +  
      \chi^{(2)}(t) \mathbf{E}(t)^2 + 
      \chi^{(3)}(t) \mathbf{E}(t)^3 + \dots \right]
    \end{equation}

    where the expansion coefficients $\chi^{(j)}(t)$ are known as the $j$th-
    order susceptibilities.

    If the applied field is weak, we find that the induced polarisation is
    proportional to that field, such that susceptibilities higher than
    $\chi^{(1)}$ are taken as zero and terms higher than the first order in
    (\ref{eqn:gen_polarisation}) are neglected. This is the regime of
    \textit{linear optics}. In this thesis we are particularly interested in
    developing and understanding numerical solutions of \textit{nonlinear}
    problems, \ie those in which higher order terms become significant. Systems
    involving weak fields are commonplace however, and as they permit analytic
    solution, along with offering insight into a broad range of optical
    phenomena, we will look at the linear regime briefly.

    The cumulative induced polarisation $P(t)$ is an integral of the
    instantaneous polarisation over all times $t'$ previous
    \begin{equation}\label{eqn:lin_pol_int}
      P(t) = \varepsilon_0 \int_{-\infty}^t{\chi(t') E(t - t') \dd t'}
    \end{equation}
    where we no longer require a superscript to define the linear susceptibility
    $\chi(t) := \chi^{(1)}(t)$. The upper limit on the integral expresses the
    causality condition that only the applied field at times in the past may
    affect the current state of the atoms.

    So far we have described atomic response in the time domain, which is
    particularly useful when looking at time-dependent input fields such as
    short pulses. But it is also instructive to look at the frequency domain,
    which is advantageous when the input field is monochromatic. The change in
    perspective is effected as usual via the Fourier
    transform.\cite{hecht2015optics} We will use the convention for the
    transform on the electric field envelope
    $\mathcal{E}(t)$
    \begin{equation}\label{eqn:fourier_transform}
      \mathcal{E}(\omega) = \int_{-\infty}^{\infty} \mathcal{E}(t) 
                              \ee^{\ii \omega t}  \dd t
    \end{equation}
    and for the inverse
    \begin{equation}\label{eqn:fourier_inv}
      \mathcal{E}(t) = \frac{1}{2\pi}\int_{-\infty}^{\infty} \mathcal{E}(\omega) 
                          \ee^{-\ii \omega t} \dd \omega
    \end{equation} 
    and we define the transform in the same way for the polarisation envelope
    $\mathcal{P}(t)$.

    Making the envelope and carrier ansatz as in (\ref{eqn:envelope_carrier})
    and substituting (\ref{eqn:fourier_inv}) into the right-hand side of
    (\ref{eqn:lin_pol_int}), we get
    \begin{equation}
      \mathcal{P}(t) = \varepsilon_0 \int_{-\infty}^{t} \chi(t') \frac{1}{2 \pi} 
      \int_{-\infty}^{\infty} \mathcal{E}(\omega) \ee^{-\ii \omega (t - t')} 
      \dd \omega \dd t'.
    \end{equation}

    We now define the frequency-dependent linear susceptibility 
    \begin{equation}
       \chi(\omega) := \int_{-\infty}^{t} \chi(t') \ee^{\ii \omega t'} \dd t'
    \end{equation}  
    such that
    \begin{equation}
      \mathcal{P}(t) = \frac{1}{2\pi} \int_{-\infty}^{\infty} \varepsilon_0 
      \chi(\omega) \mathcal{E}(\omega) \ee^{-\ii \omega t} \dd \omega.
    \end{equation}

    This expression gives the time-dependent polarisation in terms of the
    frequency components of the field weighted by that frequency-dependent
    susceptibility function. We may then take the Fourier transform of the 
    left-hand side and, as the equality holds for each frequency, we obtain the
    frequency domain linear response function
    \begin{equation}\label{eqn:lin_pol_freq} 
      \mathcal{P}(\omega) = \varepsilon_0 \chi(\omega) \mathcal{E}(\omega).
    \end{equation}

    We may now substitute this expression into (\ref{eqn:fo_mwe}), with
    $\mathcal{E}(\omega)$ time-independent by the definition
    (\ref{eqn:fourier_transform}), to obtain
    \begin{equation}\label{eqn:fo_mwe_linear}
      \frac{\partial \mathcal{E}(z, \omega)}{\partial z} = 
      \ii \frac{k}{2} \chi(\omega) \mathcal{E}(z, \omega).
    \end{equation}
    This first-order differential equation in $z$ has the analytical solution 
    \begin{equation}
      \mathcal{E}(z, \omega) = \mathcal{E}(0, \omega) \ee^{\ii \tfrac{k}{2} \chi z}.
    \end{equation}

    Having determined an expression for the electric field envelope in terms of
    the frequency-dependent susceptibility, we can put this expression for the
    envelope into \ref{eqn:fo_mwe} in order to determine the effect it will
    have. It is useful to separate the real and imaginary parts of the
    susceptibility $\chi(\omega) := \chi_R(\omega) + \ii \chi_I(\omega)$, and we
    find
    \begin{equation}\label{eqn:efield_susc_real_imag}
      \mathcal{E}(z, \omega) = 
      \mathcal{E}(0, \omega)    
      \ee^{\ii(\frac{k}{2} \chi_R z)} 
      \ee^{- \frac{k}{2} \chi_I z}.
    \end{equation}

    The real part of the frequency dependent susceptibility then corresponds to
    a phase shift $\frac{k}{2} \chi_R z$ and so dispersion, and the imaginary
    part diminishes the field. As defined in (\ref{eqn:intensity}) the intensity
    $I \propto |\mathcal{E}|^2$ and so is attenuated as it progresses through
    the medium via
    \begin{equation}\label{eqn:beer_law}
      I(z, \omega) = I_0(z, \omega) \ee^{-\alpha(\omega) z}
    \end{equation}
    where the absorption coefficient $\alpha(\omega) := k \chi_I(\omega)$. This
    is the familiar Beer law of absorption for weak fields.
    
  \subsection{Refractive Index}

    If we return back to the Maxwell equations, we see by substitution of the
    susceptibility $\chi$ into the definition for dispersion
    (\ref{eqn:elec_displacement}), we get
    \begin{equation}
        \mathbf{D} = \varepsilon_0 (1 + \chi) \mathbf{E}.
    \end{equation}

    Deriving the Maxwell wave equation again using this substitution, we find
    that the result is as for propagation in free space but with the vacuum
    speed $c$ replaced with a general phase velocity      
    \begin{equation}
        v_p = \frac{c}{\sqrt{1 + \Re \left[ \chi \right] }} = \frac{c}{n}
        \label{eqn:phase_vel_refr}
    \end{equation}
    where $n$ is the refractive index, familiar from geometrical optics. In most
    linear media $\Re \left[ \chi \right]$ is positive, so light travels more
    slowly in the medium.\cite{James1992}

    This leads us to consider what this velocity represents, considering that
    photons are massless and must travel at $c$.\cite{feynman1963feynman}. The
    resultant field wave is a superposition of the applied field wave and a
    secondary field wave which results from induced dipoles. In linear media,
    this resultant wave has the same carrier frequency but a different phase.
    The fact that the frequency is the same is the reason dense transparent
    materials exist. If the secondary wave lags the applied wave, the resultant
    wave will also lag. An observer in the medium will have to wait longer for
    the peaks of the resultant wave to come past. It is this phase difference
    which leads to an apparently slower phase velocity. The refractive index
    represents the cumulative phase difference as the light moves through the
    medium.\cite{hecht2015optics}

  
