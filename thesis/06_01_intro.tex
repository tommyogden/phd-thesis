\section{Introduction}
  \label{sec:simultons_intro}

    In previous chapters we have investigated a subset of the fascinating array
    of phenomena that have been studied in the propagation of resonant optical
    beams in three-level media. We looked at the extension of the well-known
    effect of \textsc{sit} and optical solitons to simultons in V-type media in
    chapter \ref{chp:nonlinear}, and the propagation of dark-state polaritons
    under \textsc{eit} conditions in $\Lambda$-type media in chapter
    \ref{chp:polaritons}.

    % There are other coherent phenomena such as
    % phasonium\cite{Scully1991,Eberly1996} and lasing without inversion
    % (\textsc{lwi})\cite{Blok1990,Imamoglu1989} which have also been studied
    % extensively.

    In this chapter we return our attention to the propagation of light through
    atoms in the V-type system. Our motivation for this theoretical
    investigation is recent experimental work at Durham into coherent atomic
    dynamics on the sub-nanosecond timescale in thin (on the order of a micron
    in length) vapour cells.\cite{Keaveney2013} The experiment is designed to
    investigate the effect on susceptibility of a medium with respect to a probe
    beam when significant population has been transferred into an excited state
    via a second, strong pulse.

    If the two pulses pulses were of the same wavelength, addressing the same
    transition, it would be experimentally difficult to separate detection of
    the probe signal in order to determine how its propagation had been
    affected. In addition, the decay time of the excited state (on the order of
    nanoseconds) is too short for the pulses to be well-separated in time while
    maintaining the required population transfer in the medium. Thus, a second
    laser on a separate wavelength is used for the coupling field.

    Similar systems have been designed to transfer population on the order of
    nanoseconds in rubidium vapours, using the $\Xi$ (ladder) scheme to couple an
    excited state to a higher Rydberg state.\cite{Huber2011,Baluktsian2013}
    However, the optical power required is prohibitively large due to the weak
    transition strength, so the experiment makes use again of the V-type
    configuration, coupling the ground state of rubidium to a pair of non-
    degenerate excited states.

    In the next section we shall briefly describe the experiment and present
    results, before we go on to describe the theoretical model. The experiments
    were performed and results taken by Kate A. Whittaker and James Keaveney.
    Further details of the experimental setup may be found in Keaveney,
    2013.\cite{Keaveney2013}
