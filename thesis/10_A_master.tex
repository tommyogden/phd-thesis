\chapter[Dynamics of Open Quantum Systems]
  {Dynamics of Open\\ Quantum Systems}
  \label{apx:qu_dyn}

  \section{The Density Operator}

    We typically describe the state of a quantised atomic system via its state
    vector
    \begin{equation}
      \Ket{\psi} = \sum_j{c_j} \Ket{j},
      \label{eqn:state_vector}
    \end{equation}

    a linear superposition of the eigenstates $\Ket{j}$. There are many physical
    situations, however, in which there is significant coupling to an external
    environment the state evolution of which we cannot follow.

    Atom-light interactions are one such system. While the processed of
    absorption and stimulated emission (of photons from and to the applied
    field) can be described within a \textit{closed quantum system}, the process
    of spontaneous decay due to interaction with vacuum fluctuations surrounding
    an atom cannot.

    For such \textit{open quantum systems} it is useful to generalise the
    concept of $\Ket{\psi}$ to that of the \textit{density operator} $\rho$. A
    \textit{pure state} is one that can be represented by a linear superposition
    as in (\ref{eqn:state_vector}), for which the density matrix is defined as
    $\rho = \Ket{\psi} \Bra{\psi}$. This is clearly equivalent in information to
    $\Ket{\psi}$. The usefulness of the density operator $\rho$ is that it can
    be generalised in a statistical way to represent incoherent superpositions
    of wavefunctions. We assume we have a mixture of states $\Ket{\psi_s}$ each
    with different expansions in the eigenbasis $\Ket{j}$, contained in the
    ensemble with probabilities $P(s) \ge 0$. The density operator for this
    general \textit{mixed state} is then defined as
    \begin{equation}
      \rho = \sum_s P(s) \Ket{\psi_s} \Bra{\psi_s}
    \end{equation}
    where for proper normalisation
    \begin{equation}
      \sum_s P(s) = 1.
    \end{equation}

    We may consider that $\Ket{\psi}$ describes the intrinsic Heisenberg
    uncertainty required by quantum mechanics, where $\rho$ is also able to
    describe additional uncertainty representing our state of knowledge of the
    system.\cite{Steck2007}

    The matrix representation of the density operator in a particular basis is
    also known as the density matrix. The off-diagonal elements $\rho_{jk} =
    \Bra{j} \rho \Ket{k}$ depend on relative phase of the coefficients and are
    known as \textit{coherences}. The diagonal matrix elements $\rho_{jj} =
    \Bra{j} \rho \Ket{j}$ represent the probability of a measurement finding the
    system in state $\Ket{j}$ and are known as \textit{populations}. These
    populations form a probability distribution and so must be normalised such
    that
    \begin{equation}
      \Tr\left[{\rho}\right] = \sum_j{\rho_{jj}} = 1.
    \end{equation}
    The expectation value of an operator $A$ in the density matrix formalism is
    given by
    \begin{equation}\label{eqn:density_matrix_exp_value}
      \langle A \rangle = \Tr \left[{A \rho} \right].
    \end{equation}

  \section{The Master Equation}

    By substituting the density operator $\rho$ into the standard Schr\"{o}dinger equation for motion of the quantum state
    $$
      \ii \hbar \frac{\partial}{\partial t} \ket{\psi} = \mathcal{H} \ket{\psi}
    $$
    we obtain the \textit{von Neumann equation} for unitary evolution
    \begin{equation}
      \ii \hbar \frac{\partial \rho}{\partial t} = [\mathcal{H}, \rho]
      \label{eqn:von_neu}
    \end{equation}
    which for pure states is equivalent to the Schr\"{o}dinger equation.

    We wish to extend the formalism to mixed states and derive an equation of
    motion for the open quantum system interacting with an environment. The
    observed effect of interaction with an unmonitored environment is to
    introduce non-deterministic transitions between eigenstates and dephasing
    between them.

    We start by expanding the model to include the environment, such that the
    total system is closed and described by (\ref{eqn:von_neu}). In order to
    meet this requirement, we must consider the Hilbert space of the total
    system,  and a total Hamiltonian operating over that space
    \begin{equation}
      \mathcal{H}_\Sigma = \mathcal{H} + \mathcal{H}_E + \mathcal{H}_C
    \end{equation}
    where $\mathcal{H}$ is the Hamiltonian of the system, $\mathcal{H}_E$ is the
    Hamiltonian of the environment, and $\mathcal{H}_C$ is the Hamiltonian
    describing the interaction between the system and the environment.

    As we're only concerned with the dynamics of $\mathcal{H}$, we then make a
    partial trace over the environment degrees of freedom in (\ref{eqn:von_neu})
    to obtain a master equation for time evolution of the system. The
    \textit{Lindblad master equation} is a general, trace-preserving and
    positive form for the reduced density matrix $\rho$, given by
    \begin{equation}
      \ii \hbar \frac{\partial \rho}{\partial t} = [\mathcal{H}, \rho] + 
        \mathcal{L}\left\{ \rho \right\}
      \label{eqn:lindblad_apx}
    \end{equation}
    where the Lindblad term given by
    \begin{equation}
      \mathcal{L}\left\{ \rho \right\} = \sum_j{C_j \rho C_j^\dagger - 
      \tfrac{1}{2}\left(\rho C_j^\dagger C_j + C_j^\dagger C_j \rho \right)}
      \label{eqn:lindblad_op}
    \end{equation}

    is a superoperator describing the system interaction with its environment
    via collapse operators $C_j$ coupling states. For example, to account for a
    stochastic interaction with the environment representing the decay of a
    system from state $\Ket{k}$ to state $\Ket{j}$ with rate $\Gamma_{jk}$, we
    include a collapse operator
    $$
    C_j = \sqrt{\Gamma_{jk}}\Ket{j}\Bra{k}
    $$
    in the Linblad term.

    For the Lindblad equation ($\ref{eqn:lindblad_apx}$) to be applicable as a
    master equation for the system, a couple of approximations must be
    justified. Firstly, the Born approximation requires that the environment is
    sufficiently large that it is not much affected by interaction with the
    system. We may write this as 
    \begin{equation}
      \rho_\Sigma \approx \rho \otimes \rho_E.
    \end{equation}
    Secondly, the Markov approximation requires that the time evolution depends
    on $\rho(t)$ and not any past history --- this is also called a `short-
    memory environment'.

    % An alternative approach to following the time evolution is to use a quantum-
    % jump based Monte Carlo method. This is computationally useful when the
    % density matrix is large.
