\section{Discussion}
  \label{sec:simultons_discussion}

  The results presented early on in this chapter, from experiments on a thermal
  vapour of rubidium atoms addressed by two co-propagating lasers in a V-type
  scheme, are certainly intriguing. Over a range of temperatures and powers we
  observe an early, steep peak in response of the \textsc{cw} probe on the
  \textsc{d1} transition when disturbed by a strong, short pulse on the
  \textsc{d2} transition.

  In order to understand this response behaviour, we designed a theoretical
  model for the system based on a three-level Maxwell-Bloch description for
  semiclassical propagation. We included significant physical effects of
  inhomogeneous Doppler broadening, dephasing due to collisions between moving
  atoms and hyperfine pumping to a far-detuned state in the ground state
  doublet. This simple model provides a good qualitative fit to the data over
  the range of temperatures and powers investigated experimentally, accounting
  for the peak times of the response relative to power and the relative height
  of the peak relative to atomic density.

  By considering the behaviour of a single atom addressed by the co-propagating
  fields, both in the bare atomic state basis and the coherent population
  trapping (\textsc{cpt}) basis, we gain important physical insight into the
  transient reduction of absorption in the scheme. Looking then into the effects
  of nonlinear propagation over longer distances, we determine that the
  steepening of the response is in fact due to the ability of the coupling field
  to sculpt the pulse toward a sech-shaped soliton. Over longer distances the
  simulations demonstrate that this probe soliton would propagate simultaneously
  with the resultant coupling pulse. The tendency of the pulses to separate,
  having different velocities due to the distinct absorption strengths of the
  transitions, is overcome by pulse locking.

  Notably, the area theorem as applied to simultons allows the propagation even
  of a weak probe field in this scheme through media it would ordinarily find
  opaque. Our simulations of weak probes here show such propagation.

  In single-field \textsc{sit} the field must be strong for the propagation of
  solitons to overcome the weak nonlinearity of the medium. But combining this
  approach with the \textsc{cpt} effect as described here suggests a way around
  this, and a novel approach to achieving transparent propagation of single or
  few photon pulses distinct from, but related to, both \textsc{sit} and
  \textsc{eit}.

  It should be stressed that the results of the thin cell experiment show the
  nascent formation of solitons in the \textsc{cw} probe, but for conclusive
  evidence of simulton propagation further investigation is required over longer
  distances. Balancing the requirement of low collision dephasing widths and
  high optical depth required may be difficult but we suggest is certainly worth
  attempting.

  Theoretically, for modelling few photon propagation we might wish to look at a
  quantised field method. An interesting development of the scheme might be in
  coupling to Rydberg states to introduce interactions between simultons, with
  applications in quantum information
  processing.\cite{Maxwell2013,Saffman2010,Maghrebi2015}

