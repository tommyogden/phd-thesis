\section{Thesis Structure}
  \label{sec:intro_structure}

  The remaining chapters of this thesis are structured as follows:

  \begin{description}
  \item[Chapter \ref{chp:propagation}]

    We derive a semiclassical model for propagation of light in thermal atomic
    vapours based on the Maxwell-Bloch equations. We introduce linear and
    nonlinear susceptibilities and discuss analytic results available under the
    weak probe approximation.

  \item[Chapter \ref{chp:nonlinear}]

    We take the model into the regime of nonlinear optics, demonstrating some
    effects that emerge from the interaction of strong fields with atomic
    vapours, notably self-induced transparency and simultons.

  \item[Chapter \ref{chp:polaritons}]

    We introduce the well-known phenomenon of electromagnetically induced
    transparency and the related quasiparticle known as the dark-state
    polariton. We describe how such systems may be used to store and retrieve
    light pulses.

  \item[Chapter \ref{chp:twophoton}]

    We investigate the interaction of a high-intensity beam with a thermal
    vapour of rubidium to model experimental results showing population of
    highly excited 5d states. We include angular momentum structure and
    broadening effects and consider two-photon excitation as a possible
    mechanism.

  \item[Chapter \ref{chp:simultons}]

    We present the key results of this thesis. We describe a scheme to combine
    the nonlinear phenomena of optical solitons and \textsc{eit} to propagate
    robust simultaneous pulses (simultons) in V-type media, showing excellent
    agreement with experimental results over a range of powers and temperatures.
    This scheme avoids the requirement of high-intensity pulses in the
    \textsc{sit} system, and shows that weak field soliton components may even
    be drawn from a continuous wave field.

  \item[Chapter \ref{chp:conclusions}]

    We conclude, summarise the results and suggest future directions for
    continuing the research presented.

\end{description}