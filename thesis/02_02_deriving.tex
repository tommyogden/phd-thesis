\section{Deriving the Propagation Equation}
  \label{sec:propagation_deriving}

    In this section we will derive an equation to describe the propagation of a
    classical elecromagnetic field in a polarised medium. Our starting point,
    naturally, is Maxwell's set of equations of classical electrodynamics.

  \subsection{Maxwell's Equations and the Wave Equation}

    In a medium with no free charges or free current, Maxwell's equations for
    the electric field $\mathbf{E}$ and the magnetic field $\mathbf{B}$ can be
    written\cite{jackson1998classical}
    \begin{subequations}
      \begin{align}
        \divv{\mathbf{D}} &= 0 \\
        \divv{\mathbf{B}} &= 0 \\
        \curl{\mathbf{E}} &= -\frac{\partial \mathbf{B}}{\partial t} \\
        \curl{\mathbf{B}} &= \mu_0 \frac{\partial \mathbf{D}}{\partial t}.
      \end{align}
      \label{eqn:maxwells}
    \end{subequations}
    The electric displacement $\mathbf{D}$ accounts for the effect of charges in
    a medium and is given by 
    \begin{equation}\label{eqn:elec_displacement}
      \mathbf{D} = \varepsilon_0 \mathbf{E} + \mathbf{P}
    \end{equation}
    where $\varepsilon_0$ is the vacuum permittivity and the electric
    polarisation $\mathbf{P}$ describes the cumulative effect of induced dipole
    moments in individual atoms. We'll discuss the source of polarisation in
    atoms further in section \ref{sec:propagation_susc}.

    Taking the curl of (\ref{eqn:maxwells}c) and substituting 
    (\ref{eqn:maxwells}d) we find 
    $$
      \curl{(\curl{\mathbf{E}})} = 
      -\mu_0 \frac{\partial^2 \mathbf{D}}{\partial t^2}.
    $$
    We then apply the vector identity
    $$
      \curl{(\curl{\mathbf{E}})} = 
        \nabla(\divv{\mathbf{E}}) - \nabla^2 \mathbf{E}
    $$
    and, assuming that the polarisation varies little in the plane transverse to
    propagation such that $\divv{\mathbf{P}} \approx 0$, arrive at the Maxwell
    wave equation
    \begin{equation}
      \mathbf{\nabla}^2 \mathbf{E} - \frac{1}{c^2} 
      \frac{\partial^2\mathbf{E}}{\partial t^2} = 
      \mu_0 \frac{\partial^2\mathbf{P}}{\partial t^2}.
    \label{eqn:mwe}
    \end{equation}

    The Maxwell wave equation (\textsc{mwe}) thus describes the wave-like
    propagation of an electric field $\mathbf{E}$ through a medium with
    polarisation $\mathbf{P}$. The second derivative on the right-hand side
    tells us that the accelerating charges described by $\mathbf{P}$ can act as
    a source of new components of the electromagnetic field. For this reason,
    polarisation plays a critical role in practically all optical
    phenomena.\cite{boyd2008nonlinear}

    A similar propagation equation to (\ref{eqn:mwe}) may be derived for the
    magnetic field, however we'll find that the interaction of atoms with the
    electric field dominates and so we will not consider $\mathbf{B}$ further.

    We will consider \textsc{1d} propagation along the z-axis, a restriction
    well-justified by the destructive interference of light scattered by
    atomic dipoles in directions not aligned with the incident
    radiation\cite{hecht2015optics}. Then we may write the transverse field as
    $\mathbf{E} = \hat{\mathbf{x}} E(z, t)$, where $\hat{\mathbf{x}}$ is a unit
    vector perpendicular to $z$. We assume that we are dealing with atoms that
    are electrically neutral to start with, such that all of the polarisation in
    the medium will be induced by that field, so we may similarly set
    $\mathbf{P} = \hat{\mathbf{x}} P(z, t)$. In this way we can reduce the
    \textsc{mwe} to the scalar form
    \begin{equation}
      \frac{\partial^2 E}{\partial z^2} - \frac{1}{c^2} \frac{\partial^2
        E}{\partial t^2} =    \mu_0 \frac{\partial^2 P}{\partial t^2}.
      \label{eqn:mwe_scalar} 
    \end{equation}

    In the absence of sources (\ie $P = 0$) we can solve the homogeneous
    equation (\ref{eqn:mwe_scalar}) analytically. For a nonconducting medium
    with spatially constant permeability and susceptibility the solution is a
    transverse monochromatic plane wave\cite{jackson1998classical}
    \begin{equation}
      \tilde{E}(z,t) = \tilde{E}_0 \mathrm{e}^{\mathrm{i}(k z - \omega t)}.
      \label{eqn:plane_waves}
    \end{equation}

    where as usual $k$ represents wavenumber and $\omega$ the angular frequency.
    The amplitude is given by $\tilde{E}_0$. Of course, the electric field is an
    observable physical quantity and so must be real-valued --- it is the real
    part of the complex $\tilde{E}(z,t)$ given by
    \begin{equation}
      E(z,t) = \Re[\tilde{E}(z,t)] = \tfrac{1}{2} \tilde{E}_0(z,t) 
                    \mathrm{e}^{\mathrm{i}(k z - \omega t)} 
                + \tfrac{1}{2} \tilde{E}_0^*(z,t) 
                    \mathrm{e}^{-\mathrm{i}(k z - \omega t)}.
    \end{equation}

    We continue with the complex notation for the usual reason that
    manipulations, such as adding components of different phases, are much
    simpler.

    By substituting (\ref{eqn:plane_waves}) into (\ref{eqn:mwe_scalar}) we
    obtain the simplest \textit{dispersion relation} of $\omega = c k$, \ie all
    frequencies travel at a speed $c$, a familiar result for light in vacuo.
    Such plane wave solutions will propagate without attenuation or dispersion,
    which is how we are able to look up at the night sky and observe light from
    distant stars that has travelled enormous distance through the void of
    interstellar space.

    Light travelling through matter is a more complex problem. We next want to
    look for solutions to (\ref{eqn:mwe_scalar}) in the presence of sources,
    such that we have $P \ne 0$ describing induced atomic dipoles in the medium.
    For that we will need to make a useful approximation.

  \subsection{The Slowly Varying Envelope Approximation}

    If the spectrum of the electric field is narrowband, we approximate the
    field as the product of a quasi-monochromatic \textit{carrier} function with
    angular frequency $\omega$ and wavenumber $k$ and a \textit{slowly varying
    envelope}
    \begin{equation}
      E(z,t) = \tfrac{1}{2} \mathcal{E}(z,t) 
                    \mathrm{e}^{\mathrm{i}(k z - \omega t)} 
                + \tfrac{1}{2} \mathcal{E}^*(z,t) 
                    \mathrm{e}^{-\mathrm{i}(k z - \omega t)}
    \label{eqn:envelope_carrier_ansatz}
    \end{equation}
    where the envelope is in general a complex function 
    \begin{equation}
      \mathcal{E}(z,t) = \lvert \mathcal{E} \rvert 
                            \mathrm{e}^{\mathrm{i} \varphi}.
    \end{equation}
    
    We apply the same treatment to the polarisation
    \begin{equation}
      P(z,t) = \tfrac{1}{2} \mathcal{P}(z,t) 
                    \mathrm{e}^{\mathrm{i}(k z - \omega t)} 
                + \tfrac{1}{2} \mathcal{P}^*(z,t) 
                    \mathrm{e}^{-\mathrm{i}(k z - \omega t)}
    \end{equation}
    with a polarisation envelope
    \begin{equation}
      \mathcal{P}(z,t) = \lvert \mathcal{P} \rvert 
                            \mathrm{e}^{\mathrm{i} \varphi}.
    \end{equation}

    We can substitute these into (\ref{eqn:mwe_scalar}) and by matching
    co-rotating terms we find\footnote{Here we use subscript notation: $\partial_{z} := \partial/\partial z$, $\partial_{zz} := \partial^2/\partial z^2$ \etc}
    \begin{equation}
    \left[ ( \partial_{zz} + 2 \mathrm{i} k \partial_{z} - k^2 ) - 
      \tfrac{1}{c^2} (\partial_{tt} - 
      2 \mathrm{i} \omega \partial_{t} - \omega^2 ) \right] \mathcal{E} = 
      \mu_0 (\partial_{tt} - 2 \mathrm{i} \omega \partial_{t} - 
      \omega^2) \mathcal{P}.
    \label{eqn:pre_svea}
    \end{equation}

    Now we're set to make the \textit{slowly varying envelope
    approximation}.\cite{lambropoulos2007fundamentals} If the envelope changes
    only a small amount over the distance of a wavelength, and over the duration
    of the optical period, we can take
    \begin{align*}
    \lvert \partial_{zz} \mathcal{E} \rvert &\ll k \lvert \partial_{z} 
        \mathcal{E} \rvert\\
    \lvert \partial_{tt} \mathcal{E} \rvert &\ll \omega \lvert \partial_{t} 
        \mathcal{E} \rvert
    \end{align*}
    and neglect the second derivatives. Similar approximations hold for
    $\mathcal{P}$, where we may also neglect the first-order time derivative
    \begin{align*}
      \lvert \partial_{zz} \mathcal{P} \rvert &\ll 
        k \lvert \partial_{z} \mathcal{P} \rvert \\
      \lvert \partial_{tt} \mathcal{E} \rvert &\ll 
        \omega \lvert \partial_{t} \mathcal{E} \rvert \\
      \lvert \partial_{t} \mathcal{P} \rvert &\ll 
        \omega \lvert \mathcal{P} \rvert.
    \end{align*}
    In the case of visible light with wavelengths on the order of 100 nm and
    optical periods of the order of 1 fs, where we're interested in the
    behaviour over \unit{ns} or \unit{$\mu$s}, this approximation is justified.

    With these approximations made, (\ref{eqn:pre_svea}) becomes
    \begin{equation}
      \left[ 2 \mathrm{i} k \partial_z - k^2 + \tfrac{1}{c^2} 
      (2 \mathrm{i} \omega \partial_t + \omega^2 ) \right] \mathcal{E} = 
      -\mu_0 \omega^2 \mathcal{P}
    \end{equation}
    As the carrier wavenumber and frequency are still related by the vacuum
    dispersion relation $\omega = c k$ we then obtain the first-order
    propagation equation
    \begin{equation}
      \label{eqn:fo_mwe}
      \left[ \frac{\partial}{\partial z} + \frac{1}{c} 
      \frac{\partial}{\partial t} \right] \mathcal{E} = 
      \mathrm{i} \frac{k}{2 \epsilon_0} \mathcal{P}.
    \end{equation}

    The wave equation (\ref{eqn:fo_mwe}), first-order in space $z$ and time $t$,
    can be solved numerically for a given medium, to determine how light will
    propagate through it. However, the response of the medium via $\mathcal{P}$
    may itself be affected by the field $\mathcal{E}$, so we will next look at
    how we determine $\mathcal{P}$.

  \subsection{Energy and Intensity}

    In an experiment we would not be measuring the electric field directly, but
    the energy transferred to a detector such as a photon counter. Also, any
    macroscopic measurement of light is necessarily going to extend over many
    optical cycles (of duration $\sim\!\unit[10^{-15}]{s}$). So a property
    useful to us is the time-averaged power per unit area that the
    electromagnetic field transports through the medium, which we call the
    intensity $I$ and is calculated to be\cite{griffiths1999introduction}
    \begin{equation}\label{eqn:intensity}
      I = \tfrac{1}{2}c\varepsilon_0 \lvert \mathcal{E}^2 \rvert.
    \end{equation}
